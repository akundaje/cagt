\documentclass[11pt]{article}
\usepackage{amsfonts,mathrsfs, amstext, amsmath, cancel, graphicx,subfig,fullpage}
\usepackage[all]{xy}
\begin{document}
{\center

\section*{CAGT: Clustering AGregation Tool v0.1}
\subsection*{Max Libbrecht\\\today}} \ \\ 


\section*{Introduction}
CAGT is a tool for visualizing and analyzing large set of signal profiles.  It was developed to view the histone profile neighborhoods around transcription factors.  It uses a clustering algorithm to split the data into similar sets, then uses an assortment of graphical tools to visualize them.


\section*{Installation}
To run it, you'll need rpy2 and pycluster.  You can get rpy here:
\begin{verbatim}http://rpy.sourceforge.net/rpy2.html\end{verbatim}
and pycluster here:
\begin{verbatim}http://bonsai.ims.u-tokyo.ac.jp/~mdehoon/software/cluster/software.htm\end{verbatim}
Other than that, all the other libraries used ship with python.
\\ \\
To run, type
\begin{verbatim}$ python cagt.py\end{verbatim}
This will cluster all the profiles and put a number of figures in \verb!cagt/output/!.
See below for information on configuring parameters.

\section*{Parameters}
The code takes a number of parameters, which are in the \verb!parameters.py! file in the top level folder (\verb!cagt/!).  The function of the parameters are described in the file.
%The parameters are as follows:
%\begin{itemize}
%\item \verb!filename!: Change this to the path the data file.  See below for more information about the expected file format.  
%
%\item \verb!mode!:  CAGT has two modes, ``normalize'' and ``group''.  \verb!mode! must be one of these two strings.  In normalize mode, CAGT considers only profiles with significant signal (as defined by  \verb!cutoff_quantile! and \verb!cutoff_value!), and for each 
%
%\item \verb!cutoff_quantile, cutoff_value!: If \verb!normalize! is true, the code throws out all profiles with 
%\verb!cutoff_quantile!-th quantile less than \verb!cutoff_value!.  I usually use (0.75, 2) for these.  Change \verb!cutoff_value! to 0 if you want to use  all the profiles. 
%\item \verb!num_clusters, num_passes!: \verb!num_clusters! is the ``k'' in k-medians clustering.
%The code runs k-medians \verb!num_passes! times, and picks the optimal solution out 
%of the number of passes, so decrease \verb!num_passes! to improve running time and 
%increase it to improve the clustering.
%\item label: Not used for \verb!histone_cluster!
%\end{itemize}
\\ \\
Also, the code automatically saves its progress so you don't have 
to start over if you want to restart it.  If you want to use the 
old clustering, find the lines in \verb!histone_clustering! that calls \verb!k_cluster!, 
and comment out the line where \verb!k_cluster! is called, and the line that pickles it 
The line after that unpickles the old result.  You can do the same with the line
that calls \verb!get_histone_data!.  The lines in question are well-marked in the code.



\section*{File Format}
CAGT expects a data file in the path specified by the \verb!filename! parameter in \verb!parameters.py!.  The data file should have the following (ASCII) format:
\begin{verbatim}
value<TAB>value<TAB>...<TAB>value<NEWLINE>
value<TAB>value<TAB>...<TAB>value<NEWLINE>
\end{verbatim}
There must be the same number of values per line, but there may be any number of lines.  Each line is taken to be a separate profile, with the values being the signal at each location.



%\section*{Introduction}
%hi

\end{document}